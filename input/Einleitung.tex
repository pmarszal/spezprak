Um sich mit den komplexen Themen einer Bachelorarbeit im Bereich Teilchenphysik auseinandersetzen zu können, ist ein grundlegendes Verständnis der Theorie und experimentellen Methoden der Hochenergie-Physik notwendig. Im Verlauf des Spezialisierungspraktikums wurden Themen von den Anfängen der Quantenchromodynamik (QCD), über das Standartmodell, bis hin zu \'~Beyond the Standardmodel\'~ bearbeitet. Es wurde sich intensiv damit beschäftigt wie man Vorgänge der Teilchenphysik experimentell messen kann. Im letzten Teil des Spezialisierungspraktikums, wurden fundamentale Werkzeuge für die Analyse von experimentellen Daten aus Kollisionsexperimenten (insbesondere des \lhc am \cern) behandelt.  

\\
Tadda
\\