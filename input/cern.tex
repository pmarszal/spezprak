\section{Teilchenbeschleuniger}
\subsection{Idee}
Teilchenbeschleuniger sind das notwendige Werkzeug um zu verstehen woraus Materie, auf kleinster Ebene aufgebaut ist. Möchte man ein Sandkorn auflösen so benötigt man eine Lupe um etwas zu erkennen. Um ein Bakterium aufzulösen braucht man ein Mikroskop. Um Proteine aufzulösen braucht man ein Elektronenmikroskop. Und um ein Proton aufzulösen braucht man einen Teilchenbeschleuniger. Die Tendenz geht nicht nur zu räumlich größeren Geräten, sondern zu immer kleineren Wellenlängen der Teilchen mit denen man versucht das gesuchte aufzulösen. Nach dem Prinzip der Materiewellen, benötigt man einen immer größeren Impuls. Dies wird erreicht indem man geladene Teilchen mithilfe von elektrischen Feldern beschleunigt. Die hochenergetischen Teilchen werden dann zur Kollision bebracht, wobei sie durch inelastische Stöße Einsicht in die Physik der Teilchen bieten.
\subsection{Der \lhc}
Der \lhc ist ein Proton-Proton-Collider am \cern in Genf. Er basiert auf dem Synchrotronprinzip, wonach die Magnetfelder, die den Protonstrahl auf ihrer Kreisbahn halten, abhängig von der Energie der Protonen erhöht werden. Mit einer Schwerpunktsenergie von $\sqrt{s} = 8$~TeV im Jahre 2012 und einer geplanten Schwerpunktsenergie von $13$~TeV ab 2015, ist er der hochenergetischste\comment{besseres Wort???!} Teilchenbeschleuniger der Welt. 
\section{Der \atlas-Detektor}
Der \atlas-Detektor ist einer der vier Detektoren am \lhc. 

Der innerste Detektor des \atlas ist der Pixel-Detektor. In ihm sind Schichtweise Halbleiter-Pixel verbaut, die den Zweck erfüllen die Bahn eines bei einer Kollision entstehenden Teilchens zu Messen. In ihm herrscht ein 2~T großes Magnetfeld parallel zur Strahlachse, sodass geladene Teilchen gekrümmte Bahnen, vorweisen.

Der nächstaußere Detektor ist das elektromagnetische (EM) Kalorimeter. Der Zweck des EM Kalorimeters ist die kinetischen Energien geladener Teilchen zu messen. Dazu werden die Teilchen in einem Absorbermaterial aus Blei und Stahl gebremst und das Signal wird in einer Schicht aus flüssigem Argon aufgenommen (genauere Erklärungen zur Funktionsweise in Abschnitt \ref{sec:calori}). 
Der darauffolgende Detektor ist das hadronische Kalorimeter. Dieses misst die Energie von entstehenden Teilchen, die über die starke Wechselwirkung mit Materie interagieren. Es besteht aus Stahl als Absorbermaterial und Szintillatoren als Messschichten.
Die letzte Schicht des \atlas-Detektors bilden die Myonkammern. Wie der Name bereits andeutet werden sie verwendet um Myonen, die jede andere Detektorschicht passieren können, zu detektieren und über ihre Spur ihren Impuls zu bestimmen. Sie bestehen aus Driftröhrchen (Erklärung der Funktionsweise in Abschnitt \ref{sec:drift}).

