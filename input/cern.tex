\section{Teilchenbeschleuniger}
\subsection{Idee}
Teilchenbeschleuniger sind das notwendige Werkzeug um zu verstehen woraus Materie, auf kleinster Ebene aufgebaut ist. Möchte man ein Sandkorn auflösen so benötigt man eine Lupe um etwas zu erkennen. Um ein Bakterium aufzulösen braucht man ein Mikroskop. Um Proteine aufzulösen braucht man ein Elektronenmikroskop. Und um ein Proton aufzulösen braucht man einen Teilchenbeschleuniger. Die Tendenz geht nicht nur zu räumlich größeren Geräten, sondern zu immer kleineren Wellenlängen der Teilchen mit denen man versucht das gesuchte aufzulösen. Nach dem Prinzip der Materiewellen, benötigt man einen immer größeren Impuls. Dies wird erreicht indem man geladene Teilchen mithilfe von elektrischen Feldern beschleunigt. Die hochenergetischen Teilchen werden dann zur Kollision bebracht, wobei sie durch inelastische Stöße Einsicht in die Physik der Teilchen bieten.

\subsection{Der \lhc}
Der \lhc ist ein Proton-Proton-Collider am \cern in Genf. Er basiert auf dem Synchrotronprinzip, wonach die Magnetfelder, die den Protonstrahl auf ihrer Kreisbahn halten, abhängig von der Energie der Protonen erhöht werden. Er hat einen Umfang von ca. $27$~km und liegt zwischen 45~m und 170~m unter der Erdoberfläche \cite{1748-0221-3-08-S08001}. Mit einer Schwerpunktsenergie von $\sqrt{s} = 8$~TeV im Jahre 2012 und einer geplanten Schwerpunktsenergie von $\sqrt{13}$~TeV ab 2015, ist er der hochenergetischste Teilchenbeschleuniger der Welt. Ziel des \lhc\ s ist die Suche nach dem Higgs-Boson und der Physik jenseits des Standardmodells.
Außer Protonen beschleunigt er auch Blei-Kerne, aus deren Kollsionen Erkenntnisse über die frühen Phasen der Entstehung des Universums gezogen werden können. 

