\section{Composite Service Status Identification in SOA based Infrastructures}
Um die große Menge an Daten, die am \lhc erzeugt werden, zu speichern, verwalten und auszuwerten, wird am \cern das \emph{World Wide LHC Computing Grid} verwendet. Dabei handelt es sich um ein Netzwerk von Computern und Speicherzentren, die weltweit verteilt sind. Durch die Verwendung von Grid-Computing ist es möglich Aufträge, wie z.B. Analysen von Daten, auf viele Computer aufzuteilen und damit schneller durchzuführen. Die Kommunikation zwischen einzelnen Computern und Servern basiert auf dem Prinzip der Service-Oriented-Architecture (SOA). Services erfüllen die Rolle von Aufträgen die der Benutzer dem Server erteilt. Diese sind selbst aus Sub-Services aufgebaut. Die Überwachung und Diagnose dieser Architektur ist aufgrund der zusammengesetzten Struktur nicht einfach. Hinzu kommt der Verschleiß der Hardware, der großen Einfluss auf den Zustand des Netzwerkes hat. Im Verlauf dieser Bachelorarbeit wird der Service-Zustand mithilfe von Machine-Learning-Techniken untersucht. Dazu wird die Antwortzeit der Services als Indikator für den Zustand des Systems betrachtet. Es wird erwartet, dass die Antwortzeiten der Services je nach Zustand der Infrastruktur sich um bestimmte Zentren in einem vieldimensionalen Raum anhäufen. Machine-Learning-Techniken wie Principle-Component-Analysis und Hierarchical-Clustering werden verwendet um die beste Anzahl an Clustern zu bestimmen. 
Mit der so bestimmten Anzahl an Clustern werden die Daten mit K-means- und Fuzzy-C-means-Algorithmen in diese Anzahl an Clustern eingeteilt. Die Ergebnisse werden anhand der Silhouette-methode überprüft.
Die Analysen werden anhand von Daten aus dem Speichersystem des GoeGrid Rechenzentrum betrieben werden.