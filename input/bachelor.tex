\section{Composite Service Status Identification in SOA based Infrastructures}
Um die große Menge an Daten, die am \lhc erzeugt werden, zu speichern, verwalten und auszuwerten, wird am \cern das \emph{World Wide LHC Computing Grid} verwendet. Dabei handelt es sich um ein Netzwerk von Computern und Speicherzentren, die weltweit verteilt sind. Durch die Verwendung von Grid-Computing ist es möglich Aufträge, wie z.B. Analysen von Daten, auf viele Computer aufzuteilen und damit schneller durchzuführen. Die Kommunikation zwischen einzelnen Computern und Servern basiert auf dem Prinzip der Service-Oriented-Architecture (SOA). Services erfüllen die Rolle von Aufträgen die der Benutzer dem Server erteilt. Diese sind selbst aus Sub-Services aufgebaut. 