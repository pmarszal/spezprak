\NeedsTeXFormat{LaTeX2e}[2005/12/01]
%%    2009/03/12 v1.0 GAUBM Vorlage f�r Aschlussarbeiten Physik
%% Template fuer Bachelor- und Masterarbeiten
%% an der Fakultaet fuer Physik (c) Thomas Pruschke der GA Universit�t
%% Verbesserungsvorschlaege bitte an pruschke@theorie.physik.uni-goettingen.de
%%
%% Benoetigte Pakete: datenumber
%%

%%%%%%%%%%%%%%%%%%%%%%%%%%%%%%%%%%%%%%%%%%%%%%%%%%%%%%%%%%%%%%%%%%%%%%
%%%%%%%%%% Bitte vor dem Veraendern diese Datei umbenennen! %%%%%%%%%%
%%%%%%%%%%%%%%%%%%%%%%%%%%%%%%%%%%%%%%%%%%%%%%%%%%%%%%%%%%%%%%%%%%%%%%

%% scrbook - Ersatz f�r LaTeX book Klasse aus dem KOMA Script
%% Moegliche Optionen: diejenigen der Klasse scrbook ausser titlepage

%% deutsche Arbeit:
\documentclass[bachelor,       %% Typ der Arbeit: bachelor oder master
               twoside,        %% zweiseitiges Layout
               BCOR10mm,       %% Bindekorrektur 10 mm
%               liststotoc,nomtotoc,bibtotoc, %% Aufnahme der div. Verzeichnisse
                                              %% ins Inhaltsverzeichnis
               english,ngerman, %% Alternativspr. Englisch, Dokumentspr. Deutsch
%               ngerman,english  %% Alternativspr. Deutsch, Dokumentspr. Englisch
%               final,          %% Endversion; draft fuer schnelles Kompilieren
               ]{GAUBM}

\usepackage{setspace}  %% Zur Setzung des Zeilenabstandes
\usepackage{babel}     %% Sprachen-Unterstuetzung
\usepackage{calc}      %% ermoeglicht Rechnen mit Laengen und Zaehlern
\usepackage[T1]{fontenc}       %% Unterstutzung von Umlauten etc.
\usepackage[latin1]{inputenc}  %% 
%% in aktuellem Linux & MacOS X wird standardmaessig UTF8 kodiert!
%\usepackage[utf8]{inputenc}    %% Wenn latin1 nicht geht ...

\usepackage{amsmath,amssymb} %% zusaetzliche Mathe-Symbole

\usepackage{lmodern} %% type1-taugliche CM-Schrift als Variante zur
                     %% "normalen" EC-Schrift
%% Paket fuer bibtex-Datenbanken
\usepackage[comma,numbers,sort&compress]{natbib}
\bibliographystyle{plainnat}

\newcommand{\tabheadfont}[1]{\textbf{#1}} %% Tabellenkopf in Fett
\usepackage{booktabs}                      %% Befehle fuer besseres Tabellenlayout
\usepackage{longtable}                     %% umbrechbare Tabellen
\usepackage{array}                         %% zusaetzliche Spaltenoptionen

%% umfangreiche Pakete fuer Symbole wie \micro, \ohm, \degree, \celsius etc.
\usepackage{textcomp,gensymb}

%\usepackage{SIunits} %% Korrektes Setzen von Einheiten
\usepackage{units}   %% Variante fuer Einheiten

%% Hyperlinks im Dokument; muss als eines der letzten Pakete geladen werden
\usepackage[pdfstartview=FitH,      % Oeffnen mit fit width
            breaklinks=true,        % Umbrueche in Links, nur bei pdflatex default
            bookmarksopen=true,     % aufgeklappte Bookmarks
            bookmarksnumbered=true  % Kapitelnummerierung in bookmarks
            ]{hyperref}

%% Weiter benoetigte Pakete: datenumber
%% Falls dieses Paket nicht in der Installation vorhanden ist,
%% kann es von der Seite mit diesem Template heruntergeladen werden
%% und in einem LaTeX bekanntem Verzeichnis installiert werden (notfalls
%% dem Verzeichnis mit der Arbeit).
\begin{document}
%%
%%                   Ab hier muessen die Anpassungen geschehen
%%
%% Hier den eigenen Namen einsetzen
\ThesisAuthor{Liliane}{Musterfrau}
%% Hier den Geburtsort einsetzen
\PlaceOfBirth{Dortdorf}
%% Titel Arbeit. Das erste Argument ist der deutsche, das zweite der
%% englische Titel.
\ThesisTitle{Hier steht das Thema der Arbeit in deutsch}{Here comes the title of the thesis in english}
%% Erst- und Zweitgutacher/in
%% Ist der/die Betreuer/in nicht identisch mit dem/r Erstgutachter/in,
%% muss diese/r als optionales Argument angegeben werden.
\FirstReferee[Dr.\ \ldots]{Prof.\ Dr.\ \dots}
\Institute{Institut f\"ur Theoretische Physik}
\SecondReferee{Prof.\ Dr.\ \dots}
%% Beginn und Ende des Anfertigungszeitraumes
\ThesisBegin{1}{4}{2009}
\ThesisEnd{15}{7}{2009}
%% DO NOT TOUCH THESE LINES!!!!
\frontmatter
\maketitle
\cleardoublepage
%% Zusammenfassung. Falls nicht gewuenscht, bitte auskommentieren.
\begin{abstract}
  Hier werden auf einer halben Seite die Kernaussagen der Arbeit
  zusammengefasst.
%% Optional: Stichwoerter. Wenn nicht gewuenscht, koennen die beiden
%% folgenden Zeilen geloescht werden
  \bigskip\par
  \textbf{Stichw�rter:} Physik, Bachelorarbeit
\end{abstract}
%% So laesst sich in die andere Sprache umschalten (Englisch bzw. Deutsch)
\begin{otherlanguage}{english}
\begin{abstract}
  Here the key results of the thesis can be presented in about
  half a page.
  \bigskip\par
  \textbf{Keywords:} Physics, Bachelor thesis
\end{abstract}
\end{otherlanguage}

%% Ende des Vorspanns
\cleardoublepage
%% Ab hier 1 1/2 facher Zeilenabstand (durch setspace-Paket)
\onehalfspacing
%% Erzeugt Inhaltsverzeichnis
\tableofcontents

%% Hier kann man seine Bezeichnungsweisen erklaeren. Falls nicht
%% benoetigt, bis einschliesslich \end{nomenclature} auskommentieren
\begin{nomenclature}
%% Fuer die Berechnung der Spaltenbreiten muss \usepackage{calc}
%% geladen sein!
\section*{Lateinische Buchstaben}
\noindent
\begin{longtable}[l]{p{0.2\textwidth}p{0.7\textwidth-6\tabcolsep}p{0.1\textwidth}}
  \tabheadfont{Variable}&\tabheadfont{Bedeutung}&\tabheadfont{Einheit}\\\midrule\endhead
  $A$ & Querschnittsfl"ache & $\unit{m^2}$\\
  $c$ & Geschwindigkeit & $\unitfrac{m}{s}$
\end{longtable}
\section*{Griechische Buchstaben}
\begin{longtable}[l]{p{0.2\textwidth}p{0.7\textwidth-6\tabcolsep}p{0.1\textwidth}}
  \tabheadfont{Variable}&\tabheadfont{Bedeutung}&\tabheadfont{Einheit}\\\midrule\endhead
  $\alpha$  & Winkel & $\unit{\degree}$; --\\
  $\varrho$ & Dichte & $\unitfrac{kg}{m^3}$
\end{longtable}
\section*{Indizes}
\begin{longtable}[l]{p{0.2\textwidth}p{0.8\textwidth-4\tabcolsep}}
  \tabheadfont{Index}&\tabheadfont{Bedeutung}\\\midrule\endhead
  m & Meridian\\
  $r$ & Radial
\end{longtable}
\section*{Abk"urzungen}
\begin{longtable}[l]{p{0.2\textwidth}p{0.8\textwidth-4\tabcolsep}}
  \tabheadfont{Abk"urzung}&\tabheadfont{Bedeutung}\\\midrule\endhead
  2D & zweidimensional\\
  3D & dreidimensional\\
  max & maximal
\end{longtable}
\end{nomenclature}
%% \listoftables und \listoffigures sollten nur bei genuegender Anzahl Tabellen
%% verwendet werden
%\listoffigures
%\listoftables

\mainmatter   %% Anfang Hauptteil

\chapter{Einleitung}
Diese Vorlage \verb!GAUBM!
f�r Bachelor- bzw.\ Masterarbeiten ist eine �berarbeitung der
Vorlage von Simon Dreher f�r Abschlu�arbeiten am
Institut f�r Mikrosystemtechnologie (IMTEK)
an der Universit�t Freiburg. Die eigentliche Datei mit der Klassendefinition
ist \verb!GAUBN.cls!, die Sie zusammen mit dieser Datei erhalten haben. Weitere
Dateien sind \verb!datenumber.sty! und die zugeh�rigen Sprachdefinitionen
\verb!\datenumber*.ldf!. Im Verzeichnis \verb!figures! finden sich die
von der Klasse ben�tigten Logos (Universit�t und Physik) sowie Beispielbilder
f�r die �bersetzung dieser Beispieldatei (\verb!bthesis.tex!).
Sie k�nnen diese Datei als Vorlage f�r Ihre Arbeit nutzen und entsprechend
modifizieren. Bitte denken Sie daran, sie vorher unter einem eigenen Namen
abzuspeichern.
Um die Datei anzupassen, gehen Sie wie folgt vor:

Bei den Parametern zu \verb!\documentclass[...]{GAUBM}! in der Pr�ambel
kann man durch Umschalten
zwischen \verb!english,ngerman! und \verb!ngerman,english! eine
deutsche Arbeit (erste Variante) mit Englisch als Alternativsprache bzw.\ eine
englische Arbeit (zweite Variante) mit deutsch als Alternativsprache
w�hlen. Im laufenden Text kann man mit 
\begin{verbatim}
\begin{otherlanguage}{english/ngerman}
...
\end{otherlanguage}
\end{verbatim}
zur alternativen Sprache wechseln.

Nach \verb!\begin{document}! m�ssen zuerst ein paar Befehle mit
Information �ber die Arbeit aufgerufen werden:
\begin{enumerate}
\item \verb!\ThesisAuthor{Vorname}{Nachname}!: Die Argumente sind der
Vorname und Nachname der Autorin bzw.\ des Autors der Arbeit.
\item \verb!\PlaceOfBirth{Wohnort}!: Der Geburtsort der Autorin bzw.\ des Autors.
\item \verb!\ThesisTitle{Deutscher Titel}{English title}!: Der deutsche und englische Titel der Arbeit gem�� Antrag.
\item \verb!\Institute{Institut}!: Das Institut, an dem die Arbeit angefertigt wurde.
\item \verb!\FirstReferee[Betreuer/in]{Erste/r Gutachter/in}!: Voller Titel
und Name des/r Erstgutachter/in. Ist der Betreuer der Arbeit \emph{nicht}
identisch mit dem/r Erstgutachter/in, so mu� der volle Titel und der Name
des/r Betreuer/in als optionales Argument in eckigen Klammern erscheinen.
\item \verb!\SecondReferee{Zweite/r Gutachter/in}!: Voller Titel
und Name des/r Zweitgutachter/in.
\item \verb!\ThesisBegin{Tag}{Monat}{Jahr}!: Datum des Beginns der Anfertigung
der Arbeit gem�� Antrag.
\item \verb!\ThesisEnd{Tag}{Monat}{Jahr}!: Datum der Fertigstellung der Arbeit.
\item Optional kann mit
\begin{verbatim} 
\begin{abstract}
...
\end{abstract}
\end{verbatim}
eine maximal eine halbe Seite lange Zusammenfassung eingef�gt werden.

Falls man die Zusammenfassung in der alternativen Sprache verfassen m�chte,
dann geht das mit der Befehlsfolge
\begin{verbatim} 
\begin{otherlanguage}{english/ngerman}
\begin{abstract}
...
\end{abstract}
\end{otherlanguage}
\end{verbatim}
\end{enumerate}

\chapter{Grundlagen}
In diesem Kapitel werden die theoretischen Grundlagen erl�utert.

Wichtige Gleichungen, die in der Arbeit h�ufiger zitiert werden,
sollten eine Gleichungsnummer erhalten.
\begin{equation}
  \label{eq:pythagoras}
  a^2+b^2=c^2
\end{equation}
Zum Beispiel wird in Gleichung~\ref{eq:pythagoras} der Satz des Pythagoras
angegeben.

Gerade im Bereich der Grundlagen wird viel Literatur zitiert, z.B.\
\cite{Menz97}. Falls
mehrere Literaturzitate auf einmal zitiert werden, ist folgendes
z.B.\ m�glich \cite{Horn90,DINEN6232,Menz97,Knuth84}.

\section{Unterkapitel Gliederungsebene 2}
Hier sollte etwas Text stehen.
\subsection{Unterkapitel Gliederungsebene 3}
Noch ein paar Beispiele zu Abbildungen und Tabellen:

Abbildung~\ref{fig:bildplatzhalter} verdeutlicht \dots

Wie die Abb.~\ref{fig:bildplatzhalter} und
Tab.~\ref{tab:tabellenplatzhalter} verdeutlichen \dots

\begin{figure}
  \centering
  \includegraphics[width=0.5\linewidth]{figures/bild}
  \caption{Bildbeschreibung}
  \label{fig:bildplatzhalter}
\end{figure}

Text\dots
\begin{table}
  \centering
  \begin{tabular}{llll}
    \toprule
    $A$-Wert&$B$-Wert&$C$-Wert&$D$-Wert\\
    \midrule
    aaaaaa&bbbbbbb&cccccc&ddddddd\\
    aaaaaa&bbbbbbb&cccccc&ddddddd\\
    \bottomrule
  \end{tabular}
  \caption{Tabellenbeschreibung}
  \label{tab:tabellenplatzhalter}
\end{table}

Text\dots

\chapter{Experimentelle Vorgehensweise}
Text\dots
\chapter{Ergebnisse}
Text\dots
\chapter{Diskussion}
Text\dots
\section{Unterkapitel}
Text\dots
\subsection{Unterkapitel}
Text\dots
\chapter{Zusammenfassung}
Text\dots

\appendix
\chapter{erster Anhang}
Text\dots
\chapter{zweiter Anhang}
Text\dots

\cleardoublepage
%% Bibliographie. Das Argument muss der Name der BIBTeX-Datenbank stehen.
%% Ein Beispiel fuer eine solche Datenbank finden Sie in bthesis_datenbank.bib
\bibliography{bthesis_datenbank} 

\chapter*{Danksagung}
Dank\dots

%% Dieser Befehl MUSS am Ende stehen und erzeugt die Erklaerung ueber die
%% benutzten Mittel
\Declaration
\end{document}
