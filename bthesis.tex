\NeedsTeXFormat{LaTeX2e}[2005/12/01]
%%    2009/03/12 v1.0 GAUBM Vorlage fuer Abschlussarbeiten Physik
%% Template fuer Bachelor- und Masterarbeiten
%% an der Fakultaet fuer Physik (c) Thomas Pruschke der GA Universitaet
%% Verbesserungsvorschlaege bitte an pruschke@theorie.physik.uni-goettingen.de
%%
%% Benoetigte Pakete: datenumber
%%

%%%%%%%%%%%%%%%%%%%%%%%%%%%%%%%%%%%%%%%%%%%%%%%%%%%%%%%%%%%%%%%%%%%%%%
%%%%%%%%%% Bitte vor dem Veraendern diese Datei umbenennen! %%%%%%%%%%
%%%%%%%%%%%%%%%%%%%%%%%%%%%%%%%%%%%%%%%%%%%%%%%%%%%%%%%%%%%%%%%%%%%%%%

%% scrbook - Ersatz fuerr LaTeX book Klasse aus dem KOMA Script
%% Moegliche Optionen: diejenigen der Klasse scrbook ausser titlepage
%% Updates, Fixes und Modifikationen von Boris Lemmer, 07.01.2015

%% deutsche Arbeit:
\documentclass[bachelor,       %% Typ der Arbeit: bachelor oder master
               twoside,        %% zweiseitiges Layout
               BCOR10mm,       %% Bindekorrektur 10 mm
%               liststotoc,nomtotoc,bibtotoc, %% Aufnahme der div. Verzeichnisse
                                              %% ins Inhaltsverzeichnis
%               english,ngerman, %% Alternativspr. Englisch, Dokumentspr. Deutsch
               ngerman,english  %% Alternativspr. Deutsch, Dokumentspr. Englisch
%               final,          %% Endversion; draft fuer schnelles Kompilieren
               ]{GAUBM}

\usepackage{setspace}  %% Zur Setzung des Zeilenabstandes
\usepackage{babel}     %% Sprachen-Unterstuetzung
\usepackage{calc}      %% ermoeglicht Rechnen mit Laengen und Zaehlern
\usepackage[T1]{fontenc}       %% Unterstutzung von Umlauten etc.
\usepackage[latin1]{inputenc}  %% 
%% in aktuellem Linux & MacOS X wird standardmaessig UTF8 kodiert!
%\usepackage[utf8]{inputenc}    %% Wenn latin1 nicht geht ...

\usepackage{amsmath,amssymb} %% zusaetzliche Mathe-Symbole

\usepackage{lmodern} %% type1-taugliche CM-Schrift als Variante zur
                     %% "normalen" EC-Schrift
%% Paket fuer bibtex-Datenbanken
\usepackage[comma,numbers,sort&compress]{natbib}
%% modified by A.Quadt, 01.09.2010
% \bibliographystyle{plainnat}
\bibliographystyle{bthesis}
% added by A.Quadt, 01.09.2010
%\documentclass[11pt, a4paper, headsepline, footsepline]{scrbook}

\usepackage[latin1]{inputenc}
\usepackage{longtable}
\usepackage[it, bf]{caption}
\usepackage[dvips]{graphicx}
\usepackage{amsfonts}
\usepackage{amsmath}
\usepackage{mathrsfs}
\usepackage{epsfig}
\usepackage[clearempty]{titlesec}
\usepackage{booktabs}
\usepackage{hhline}
\usepackage{array}
\usepackage{subfigure}
%\usepackage{floatflt}
\usepackage{graphicx}% Include figure files
\usepackage{dcolumn}% Align table columns on decimal point
\usepackage{bm}% bold math
\usepackage{mathrsfs} 
\usepackage{amssymb}

% added by A.Quadt, 07.09.2010
\usepackage{amsfonts}
\usepackage{amsmath}
\usepackage{mathrsfs}
\usepackage{xspace}
%

\setlength{\oddsidemargin}{0cm}
\setlength{\evensidemargin}{0cm}
\setlength{\topmargin}{-1cm}
\setlength{\textheight}{23cm}
\setlength{\textwidth}{16cm}

\pagestyle{headings}

\renewcommand{\sectfont}{\bfseries\rmfamily}
\renewcommand{\floatpagefraction}{0.7}
\renewcommand{\textfraction}{0.1}

% abbreviations added, A.Quadt, 07.09.2010
%
\newcommand{\dzero}      {D\O\xspace}
\newcommand{\cdf}        {CDF\xspace}
\newcommand{\uubar}      {\mbox{$u\bar{u}$}}
\newcommand{\ddbar}      {\mbox{$d\bar{d}$}}
\newcommand{\ccbar}      {\mbox{$c\bar{c}$}}
\newcommand{\ssbar}      {\mbox{$s\bar{s}$}}
\newcommand{\ttbar}      {\mbox{$t\bar{t}$}}
\newcommand{\bbbar}      {\mbox{$b\bar{b}$}}
\newcommand{\wjets}      {\mbox{$W + 4\; jets$}}
\newcommand{\pttbar}     {\mbox{$p_{t\bar{t}}$}}
\newcommand{\pwjets}     {\mbox{$p_{W +4 \; jets}$}}
\newcommand{\ljets}      {\mbox{$\ell + \; jets$}}
\newcommand{\ejets}      {\mbox{$e + \; jets$}}
\newcommand{\mujets}     {\mbox{$\mu + \; jets$}}
%
%
\newcommand{\fermilab}  {{F{\sc ermilab}}\xspace}
\newcommand{\tevatron}  {{T{\sc evatron}}\xspace}
\newcommand{\opal}      {{O{\sc pal}}\xspace}
\newcommand{\cern}      {{C{\sc ern}}\xspace}
\newcommand{\fnal}      {{F{\sc nal}}\xspace}
\newcommand{\atlas}     {{A{\sc tlas}}\xspace}
\newcommand{\lhc}       {{L{\sc hc}}\xspace}
\newcommand{\lep}       {{L{\sc ep}}\xspace}
\newcommand{\slc}       {{S{\sc lc}}\xspace}
\newcommand{\pep}       {{P{\sc ep}}\xspace}
\newcommand{\petra}     {{P{\sc etra}}\xspace}
\newcommand{\hera}      {{H{\sc era}}\xspace}
\newcommand{\lepaleph}  {{A{\sc leph}}\xspace}
\newcommand{\delphi}    {{D{\sc elphi}}\xspace}
\newcommand{\leplthree} {{L{\sc 3}}\xspace}
\newcommand{\lepopal}   {{O{\sc pal}}\xspace}
\newcommand{\doris}     {{D{\sc oris}}\xspace}
\newcommand{\isr}       {{I{\sc sr}}\xspace}
\newcommand{\desy}      {{D{\sc esy}}\xspace}
\newcommand{\kek}       {{K{\sc ek}}\xspace}
\newcommand{\slac}      {{S{\sc lac}}\xspace}
\newcommand{\tristan}   {{T{\sc ristan}}\xspace}
\newcommand{\cms}       {{C{\sc ms}}\xspace}
\newcommand{\alice}     {{A{\sc lice}}\xspace}
\newcommand{\zeus}      {{Z{\sc eus}}\xspace}
\newcommand{\hone}      {{H{\sc 1}}\xspace}
\newcommand{\minuit}    {{M{\sc inuit}}\xspace}
\newcommand{\herwig}    {{H\sc{erwig}}\xspace}
\newcommand{\acermc}    {{A\sc{cerMC}}\xspace}
\newcommand{\evtgen}    {{E\sc{vtgen}}\xspace}
\newcommand{\mcfm}      {{M\sc{cfm}}\xspace}
\newcommand{\mcatnlo}   {{M\sc{c@nlo}}\xspace}
\newcommand{\sherpa}    {{S\sc{herpa}}\xspace}
\newcommand{\jimmy}     {{J\sc{immy}}\xspace}
\newcommand{\cteq}      {{C\sc{teq}}\xspace}
\newcommand{\pythia}    {{P\sc{ythia}}\xspace}
\newcommand{\jetnet}    {{J\sc{etnet}}\xspace}
\newcommand{\isajet}    {{I\sc{sajet}}\xspace}
\newcommand{\jetset}    {{J\sc{etset}}\xspace}
\newcommand{\vecbos}    {{V\sc{ecbos}}\xspace}
\newcommand{\alpgen}    {{A\sc{lpgen}}\xspace}
\newcommand{\vegas}     {{V\sc{egas}}\xspace}
\newcommand{\gnu}       {{G\sc{nu}}\xspace}
\newcommand{\onetop}    {{O\sc{neTop}}\xspace}
\newcommand{\ztop}      {{Z\sc{Top}}\xspace}
\newcommand{\toprex}    {{T\sc{opRex}}\xspace}
\newcommand{\singletop} {{S\sc{ingleTop}}\xspace}
\newcommand{\madgraph}  {{M\sc{adgraph}}\xspace}
\newcommand{\madevent}  {{M\sc{adevent}}\xspace}
\newcommand{\comphep}   {{C\sc{omphep}}\xspace}
\newcommand{\qq}        {{Q\sc{q}}\xspace}
\newcommand{\tauola}    {{T\sc{auola}}\xspace}
\newcommand{\geant}     {{G\sc{eant}}\xspace}
\newcommand{\GEANT}     {{G\sc{eant}}\xspace}
\newcommand{\amegic}    {{A\sc{megic++}}\xspace}

\newcommand{\met}       {\mbox{$\not\!\!E_T$}\xspace}
\newcommand{\metcal}    {\mbox{$\not\!\!E_{Tcal}$}\xspace}
\newcommand{\MET}       {$\not\!\!E_T$}
\newcommand{\lowmet}    {low-\mbox{$\not\!\!E_T$}-QCD\xspace}

\newcommand{\runi}      {Run~I\xspace}
\newcommand{\runii}     {Run~II\xspace}

%\newcommand{\dzero}      {D\O}
%\newcommand{\uubar}      {\mbox{$u\bar{u}$}}
%\newcommand{\ddbar}      {\mbox{$d\bar{d}$}}
%\newcommand{\ccbar}      {\mbox{$c\bar{c}$}}
%\newcommand{\ssbar}      {\mbox{$s\bar{s}$}}
%\newcommand{\ttbar}      {\mbox{$t\bar{t}$}}
%\newcommand{\bbbar}      {\mbox{$b\bar{b}$}}
%\newcommand{\wjets}      {\mbox{$W + 4\; jets$}}
%\newcommand{\pttbar}     {\mbox{$p_{t\bar{t}}$}}
%\newcommand{\pwjets}     {\mbox{$p_{W +4 \; jets}$}}
%\newcommand{\ljets}      {\mbox{$l + \; jets$}}
%\newcommand{\ejets}      {\mbox{$e + \; jets$}}
%\newcommand{\mujets}     {\mbox{$\mu + \; jets$}}
%\newcommand{\herwig}     {{\sc herwig}}



\newcommand{\tabheadfont}[1]{\textbf{#1}} %% Tabellenkopf in Fett
\usepackage{booktabs}                      %% Befehle fuer besseres Tabellenlayout
\usepackage{longtable}                     %% umbrechbare Tabellen
\usepackage{array}                         %% zusaetzliche Spaltenoptionen

%% umfangreiche Pakete fuer Symbole wie \micro, \ohm, \degree, \celsius etc.
\usepackage{textcomp,gensymb}

%\usepackage{SIunits} %% Korrektes Setzen von Einheiten
\usepackage{units}   %% Variante fuer Einheiten

%% Hyperlinks im Dokument; muss als eines der letzten Pakete geladen werden
\usepackage[pdfstartview=FitH,      % Oeffnen mit fit width
            breaklinks=true,        % Umbrueche in Links, nur bei pdflatex default
            bookmarksopen=true,     % aufgeklappte Bookmarks
            bookmarksnumbered=true  % Kapitelnummerierung in bookmarks
            ]{hyperref}

%% Weiter benoetigte Pakete: datenumber
%% Falls dieses Paket nicht in der Installation vorhanden ist,
%% kann es von der Seite mit diesem Template heruntergeladen werden
%% und in einem LaTeX bekanntem Verzeichnis installiert werden (notfalls
%% dem Verzeichnis mit der Arbeit).
\begin{document}
%%
%%                   Ab hier muessen die Anpassungen geschehen
%%
%% Hier den eigenen Namen einsetzen
\ThesisAuthor{Philip}{Marszal}
%% Hier den Geburtsort einsetzen
\PlaceOfBirth{Kassel}
%% Titel Arbeit. Das erste Argument ist der deutsche, das zweite der
%% englische Titel.
\ThesisTitle{Spezialisierungspraktikum in der Kern- und Teilchenphysik}{}
%% Erst- und Zweitgutacher/in
%% Ist der/die Betreuer/in nicht identisch mit dem/r Erstgutachter/in,
%% muss diese/r als optionales Argument angegeben werden.
%% Diese Angaben beziehen sich auf Institut-externe BetreuerInnen und sollten nur in Ausnahmen relevant sein.
%%\FirstReferee[Dr.\ \ldots]{Prof.\ Dr.\ \dots} % fuer externe Betreung
\FirstReferee{Prof.\ Dr.\ \dots}                % fuer interne Betreung
%% Optionen mit Stand 01. Januar 2014:
%% Prof.~Dr.~Ariane Frey
%% Priv.Doz.~Dr.~J{\"o}rn Gro{\ss}e-Knetter
%% Prof.~Dr.~Hans Hofs{\"a}ss
%% Prof.~Dr.~Arnulf Quadt
%% Jun.Prof.~ Steffen Schumann
\Institute{II. Physikalischen Institut}
\SecondReferee{Prof.\ Dr.\ \dots}
%% added by A.Quadt, 31.08.2010
%% Referenz Nummer der Bachelorarbeit im Institut
\ReferenceNumber{II.Physik-UniG{\"o}-BSc-2014/01}
%%
%% Beginn und Ende des Anfertigungszeitraumes
\ThesisBegin{1}{4}{2014}
\ThesisEnd{15}{7}{2014}
%% DO NOT TOUCH THESE LINES!!!!
\frontmatter
\maketitle
\cleardoublepage
%% Zusammenfassung. Falls nicht gewuenscht, bitte auskommentieren.
\begin{otherlanguage}{english} % Boris Lemmer, 07.01.2015
\begin{abstract}
  Hier werden auf einer halben Seite die Kernaussagen der Arbeit
  zusammengefasst.
%% Optional: Stichwoerter. Wenn nicht gewuenscht, koennen die beiden
%% folgenden Zeilen geloescht werden
  \bigskip\par
  \textbf{Stichw{\"o}rter:} Physik, Bachelorarbeit
\end{abstract}
\end{otherlanguage}
%% So laesst sich in die andere Sprache umschalten (Englisch bzw. Deutsch)
\begin{otherlanguage}{english} % Boris Lemmer, 07.01.2015
\begin{abstract}
  Here the key results of the thesis can be presented in about
  half a page.
  \bigskip\par
  \textbf{Keywords:} Physics, Bachelor thesis
\end{abstract}
\end{otherlanguage}

%% Ende des Vorspanns
\cleardoublepage
%% Ab hier 1 1/2 facher Zeilenabstand (durch setspace-Paket)
\onehalfspacing
%% Erzeugt Inhaltsverzeichnis
\tableofcontents

%% Hier kann man seine Bezeichnungsweisen erklaeren. Falls nicht
%% benoetigt, bis einschliesslich \end{nomenclature} auskommentieren
\begin{nomenclature}
%% Fuer die Berechnung der Spaltenbreiten muss \usepackage{calc}
%% geladen sein!
\section*{Lateinische Buchstaben}
\noindent
\begin{longtable}[l]{p{0.2\textwidth}p{0.7\textwidth-6\tabcolsep}p{0.1\textwidth}}
  \tabheadfont{Variable}&\tabheadfont{Bedeutung}&\tabheadfont{Einheit}\\\midrule\endhead
  $A$ & Querschnittsfl{\"a}che & $\unit{m^2}$\\
  $c$ & Geschwindigkeit & $\unitfrac{m}{s}$
\end{longtable}
\section*{Griechische Buchstaben}
\begin{longtable}[l]{p{0.2\textwidth}p{0.7\textwidth-6\tabcolsep}p{0.1\textwidth}}
  \tabheadfont{Variable}&\tabheadfont{Bedeutung}&\tabheadfont{Einheit}\\\midrule\endhead
  $\alpha$  & Winkel & $\unit{\degree}$; --\\
  $\varrho$ & Dichte & $\unitfrac{kg}{m^3}$
\end{longtable}
\section*{Indizes}
\begin{longtable}[l]{p{0.2\textwidth}p{0.8\textwidth-4\tabcolsep}}
  \tabheadfont{Index}&\tabheadfont{Bedeutung}\\\midrule\endhead
  m & Meridian\\
  $r$ & Radial
\end{longtable}
\section*{Abk{\"u}rzungen}
\begin{longtable}[l]{p{0.2\textwidth}p{0.8\textwidth-4\tabcolsep}}
  \tabheadfont{Abk"urzung}&\tabheadfont{Bedeutung}\\\midrule\endhead
  2D & zweidimensional\\
  3D & dreidimensional\\
  max & maximal
\end{longtable}
\end{nomenclature}
%% \listoftables und \listoffigures sollten nur bei genuegender Anzahl Tabellen
%% verwendet werden
%\listoffigures
%\listoftables

\mainmatter   %% Anfang Hauptteil

\chapter{Einleitung}
\chapter{Grundlagen}
\chapter{Experimentelle Vorgehensweise}
\chapter{Hinweise zum Formulieren und Zitieren}
\chapter{Ergebnisse}
\chapter{Diskussion}
\chapter{Zusammenfassung}

\appendix
\chapter{Anhang}


\cleardoublepage
%% Bibliographie. Das Argument muss der Name der BIBTeX-Datenbank stehen.
%% Ein Beispiel fuer eine solche Datenbank finden Sie in bthesis_datenbank.bib
\bibliography{bthesis_datenbank} 


%% Dieser Befehl MUSS am Ende stehen und erzeugt die Erklaerung ueber die
%% benutzten Mittel
\begin{otherlanguage}{ngerman}
\Declaration
\end{otherlanguage}
\end{document}
